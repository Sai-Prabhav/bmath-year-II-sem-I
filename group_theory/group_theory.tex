\documentclass{book}
\usepackage{../notes}
\usepackage{makeidx}

\title{Group Theory}
\author{V Sai Prabhav}
\date{ISI Year II, Semester I}

\makeindex

\begin{document}

\maketitle


\chapter{Introduction to Sets}
Set Theory, there is a symbol \('\epsilon'\) called belongs to. A statement in set theory can look like this \(x \epsilon y\).

\begin{axiom}
  There exist a set with no elements.
\end{axiom}
\begin{axiom}
  Two sets with same elements are equal
\end{axiom}
\begin{definition}
  \(A  \subseteq B\) if all elements of \(A\) are in \(B\)
\end{definition}

\begin{axiom}
  if \(X,Y\) are sets then \(\left\{ x,y  \right\} \)
\end{axiom}
\begin{axiom}
  \(A\) is a set then \( union  A = \left\{x \in y | y \in A \right\} \)
\end{axiom}
\begin{axiom}[Power set]
  Given a set A there exist \(P(A)= \left\{ B \mid \fa B \subseteq A  \right\} \)
\end{axiom}
\begin{axiom}[Infinite axiom]
  There exist I \( \phi \in I \and \fa y \in I (P(y) \in I)\)
\end{axiom}
\begin{axiom}
  A, B are sets then \(A \times B = \{(x,y) \mid x \in A , y \in B \}\) is a set.

\end{axiom}
\begin{definition}[Relation]
  Relation R on a set ea is a subset \(R \subseteq A \times A\)
  \[
    (x,y) \in R \then x R y
  \]
\end{definition}


\begin{axiom}[Axiom of choice]\label{axi:choice}
  Let A be a collection of non empty disjoint sets then there exist a set C consisting of exactly one element from each set in A.
\end{axiom}
\begin{definition}
  A relation R is called reflexive if \(xRx \fa x \in A\).

  A relation R is called symmetric if \(xRy \then yRx\).

  A relation R is called anti-symmetric if \(xRy , yRy \then y=x\).

  A relation R is called transitive if \(xRy, yRz \then xRz \).

\end{definition}
\begin{definition}[Partial order]
  A partial order on set \(A\) is a reflexive transitive anti-symmetric relation.
\end{definition}

\begin{definition}[Total order]
  Total order R on a set A is a partial order \(\st \fa x,y \in A\) either \(xRy \text{ or } yRx \)
\end{definition}

\begin{definition}[Minimal element]
  A total order on a set A is called a well ordered if given any non empty subset B of A.
\end{definition}


\begin{theorem}[Well-ordering Principle]
  Every set can be well ordered. (It is equvivalent to \autoref{axi:choice}  )
\end{theorem}

\begin{lemma}[Zorn's Lemma]
  Let \(A\) be a set and \( \le \) be a partial order on \(A \st \)
  every chain in \(A\) has an upper bound in \(A\) then \(A \) has a maximal element.
\end{lemma}
\begin{definition}[Chain]
  A chain in \(A\) is a subset which is a totally ordered with respect to \(<\)
\end{definition}
\begin{definition}[Upper Bound]
  Let \(C \subseteq A \). \(x \in A \) is an upper bound of \(C\) if \(\fa y \in C y < x\).
\end{definition}
\begin{definition}[Maximal element]
  \(x \in A\) is called a maximal element if \( \fa y \in A x<y  \implies x=y\)
\end{definition}
\begin{theorem}
  TFAE
  \begin{itemize}
    \item Axiom of choice
    \item Well- ordering principle
    \item Zorin's lemma
  \end{itemize}
\end{theorem}

\begin{definition}
  A relation R is called an equivalence relation if it is reflexive ,symmetric and transitive.
\end{definition}
\begin{definition}[Equivalence class]
  Let \(x \in A\) then \([x]= \left\{ yRx \mid y \in A   \right\} \subseteq \) called the equivalence class of x.
\end{definition}
\begin{definition}
  \(\bigcup_{x \in A}[x]= A \) and for \(x,y \in A \ [x] \cap [y] = \phi  \text{ or } [x]=[y]\). That is we get a partition of A.
\end{definition}
\begin{axiom}
  I be an indexing set. Then \( \times _{i \in I}A_i = \left\{ f: I \to UA_i \mid \st f(i) \in A_i \right\}    \)
\end{axiom}
\begin{definition}[Principle of Induction]
  let \(S(n), n \in \NN \) be statements. Suppose \(S(1  )\) is true and \(S(n)  \implies S(n+1) \then S(n) \) is true for all \(n\).
\end{definition}
\begin{definition}[Transfinite Induction]
  Let I be a well ordered set and \(S(i)\) be a statement for \(\fa i \in I\). Suppose \(s(j) \fa j < i  \implies s(i)\) then \(S(i)\) is true \(\fa i \in I\)
\end{definition}

\begin{definition}[Group]
  A group is a triple \((G,.,e)\) where G is a set, \(.\) is a binary operator on G and \(e \in G\) satisfying the following axioms.
  \begin{itemize}
    \item for \(a,b,c \in G  (a \cdot b).c = a\cdot(b\cdot c)\).
    \item \(a \cdot e =  e \cdot a = a \fa a \in G\).
    \item \(\fa a \in G \ \exists \ b \in G \st a \cdot b = b \cdot a= e \).
  \end{itemize}

\end{definition}

\begin{definition}
  \(G, \cdot , e \) is called abelian if \(\fa a,b \in G a \cdot b = b \cdot a\)
\end{definition}

\section{Binary Operations}
\section{Groups and Examples}
\section{Elementary Properties of Groups}

\chapter{Subgroups and Cosets}
\section{Subgroups}
\section{Cyclic Groups}
\section{Cosets and Lagrange's Theorem}

\chapter{Homomorphisms and Isomorphisms}
\section{Homomorphisms}
\section{Kernel and Image}
\section{Isomorphism Theorems}

\chapter{Group Actions}
\section{Group Actions and Permutation Representations}
\section{Orbits and Stabilizers}
\section{The Class Equation}




\printindex

\end{document}
