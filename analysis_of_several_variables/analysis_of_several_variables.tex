\documentclass{book}
\usepackage{../notes}
\usepackage{makeidx}

\title{Analysis of Several Variables}
\author{V Sai Prabhav}
\date{ISI Year II, Semester I}

\makeindex

\begin{document}

\maketitle

\chapter{Topology of Euclidean Space}

\section{Introduction}
\begin{definition}
  \[
    \RR ^{n} = \RR \times \cdots \times \RR = \left\{(x_{1} , x_{2} , \cdots , x_n)| x_{1} ,\cdots,x_n \in \RR \right\}
  \]
\end{definition}
\section{\(\RR ^{n} \) as a Vector Space}

\begin{itemize}
  \item \(\left<x,y\right> = \sum_{i=1}^{n}  \fa x,y= \RR ^{n} \)
  \item \(\left\{e_i\right\}_{i=1} ^{n} \) is an orthonormal basis for \(\RR ^{n} \)
  \item Simplest maps \(\RR^n \to \RR ^{m} \)?
        \bf{Linear maps}: It sends lines to lines
\end{itemize}
\begin{example}
  Linear map \(f:\RR \to \RR\)
  \[
    f(x)= x f(1) \fa x \in \RR
  \]
\end{example}
\begin{corollary}
  if \(c \in \RR \), then \(x \mapsto cx \) is a liner map

  \(therfore \left\{f:\RR roundback \text{Linear}\right\}\) \iff \(\RR \)

\end{corollary}
\begin{remark}
  Let \(l:\RR ^{n}  -> \RR ^{m} \) be a linear map.
  \[
    le_j= \sum_{i=1}^{m} a _{ij} e_i \fa j=1,\cdots ,n
  \]
  we write \(\left[L\right] _{\left\{e_i\right\}^{n} _{i=1} }^{\left\{e_j}\right\}^{m_{j-1} }= \left(a_{ij}\right)_{m \times n}  \)
\end{remark}
\begin{definition}
  Define distance function \(d:\RR ^n \times \RR ^{n} \to \RR  \ge 0\) by \(d(x,y)=\left||x-y|\right|\)
\end{definition}
\begin{definition}[Scalar Product]

  \[
  \]
\end{definition}
\begin{remark}
  \begin{itemize}
    \item \(\left|x\right|= \left<x,y\right>^{1/2} \)
    \item \(<,> \) is liner w.r.t 1st and second slot
  \end{itemize}
\end{remark}

\begin{theorem}[Cauchy Schwarz inequality ]
  \[
    \fa x,y \in \RR ^{n} , <x,y> \le \left|x\right|\left|y\right|
  \]
\end{theorem}
\begin{proof}
  \[
    \begin{aligned}
      0\le \sum_{i=1}^{n}       \sum_{j=1}^{n} \left(x_iy_j-x_jy_i\right)^{2}
                     & = 2 \left[ \sum_{}^{} \sum_{ }^{}  x_i ^{2} y_i ^{2} - \sum_{i,j}^{} x_ix_jy_iy_j\right] \\
                     & =\left[ \left|x\right|^{2} \left|y\right|^{2} - <x,y> ^{2} \right]

      \implies <x,y> & \le \left|x\right|\left|y\right|
    \end{aligned}
  \]
\end{proof}
\begin{remark}
  equality occurs if and only if \(x_iy_j=x_jy_i \fa i,j\)
\end{remark}
\begin{theorem}
  Let \(L: \RR ^{n} \to \RR^m \) a linear map. Then \(\exists M > 0 \st \left|Lx\right|\leq M \left|x\right| \fa x \in \RR ^{n} \)
\end{theorem}
\begin{proof}
  As \(x= \sum_{i=1}^{n} x_ie_i,\)

  so \(Lx= \sum_{i=1}^{n} x_i L e_i\)
  \[
    \implies \left|Lx\right| = \left|\sum_{i=1}^{n} x_i L e_i \right| \le \sum_{i=1}^{n} \left|x_iLe_i\right|= \sum_{}^{} \left|x_i\right|\left|Le_i\right|\le \left|x\right| \left(\left|Le_i\right|\right)
  \]
  \[
    \left< x,y ^{2} ^{2}  \right>
  \]
\end{proof}

\section{Connectedness}
\[
  \circlearrowleft

  \norm{e^{e^{e^{e^{e^{e}}}}}}
  \norm{x ^{2} \frac{x}{e^{e^e^{e^{e^{e^{e^{e^{e^{e^{e^{e  }}}}}}}}}{}}}}
\]
\chapter{Differentiation}
\section{The Derivative as a Linear Map}
\section{The Chain Rule}
\section{Inverse and Implicit Function Theorems}

\chapter{Integration}
\section{Multiple Integrals}
\section{Change of Variables}
\section{Fubini's Theorem}

\chapter{Manifolds}
\section{Submanifolds of Euclidean Space}
\section{Tangent Spaces}
\section{Differential Forms}

\printindex

\end{document}
